\section{Autre partie}
\label{chap:autre_partie}
% [Votre texte ici]
%\lipsum[1]
\subsection{Partie 1}
\label{sec:autre_partie_1}

\subsubsection{Sous-partie 1}
\label{ssec:autre_partie_1_1}
% Pour citer une source, utilisez la commande \cite{cle_bibtex}.
% Par exemple, les travaux de LeCun et al. ont été fondamentaux dans le domaine du Deep Learning \cite{lecun2015deep}.
Le Deep Learning est un domaine de l'intelligence artificielle qui a connu une croissance exponentielle, comme décrit dans l'ouvrage de référence de Goodfellow et al. \cite{goodfellow2016deep}. 
De nombreux outils open-source, tels que TensorFlow \cite{tensorflow2022}, ont rendu ces technologies accessibles.

\subsubsection{Sous-partie 2}
\label{ssec:autre_partie_1_2}
% [Votre texte ici]
%\lipsum[2]
\paragraph{Sous-sous-partie 1}
\label{sssec:autre_partie_1_2_1}
% [Votre texte ici]
%\lipsum[3]

\paragraph{Sous-sous-partie 2}
\label{sssec:autre_partie_1_2_2}
% [Votre texte ici]
%\lipsum[4]
\subparagraph{Paragraphe 1 (agissant comme titre niveau 5)}
% [Votre texte ici]
%\lipsum[5]

\subparagraph{Paragraphe 2}
% [Votre texte ici]
%\lipsum[6]

\paragraph{Sous-sous-partie 3}
\label{sssec:autre_partie_1_2_3}
% [Votre texte ici]
%\lipsum[7]

\subsection{Partie 2}
\label{sec:autre_partie_2}
% [Votre texte ici]
%\lipsum[8]
\subsubsection{Sous-partie 1}
\label{ssec:autre_partie_2_1}
% [Votre texte ici]
%\lipsum[9]

\subsubsection{Sous-partie 2}
\label{ssec:autre_partie_2_2}
% [Votre texte ici]
%\lipsum[10]

\subsubsection{Sous-partie 3}
\label{ssec:autre_partie_2_3}
% [Votre texte ici]
%\lipsum[11] 