\section{Présentation du projet}
\label{chap:presentation_projet}
% Le texte ci-dessous est un exemple de remplissage.
% Décommentez la ligne ci-dessous et remplacez le texte pour ajouter votre contenu.
% [Votre texte ici]
%\lipsum[1]

\subsection{Sujet}
\label{sec:sujet}
% Contenu sur le sujet du projet.
% [Votre texte ici]
%\lipsum[3]

\subsection{Problématique soulevée}
\label{sec:problematique}
% Description de la problématique.
% [Votre texte ici]
%\lipsum[4]

\begin{figure}[H]
    \centering
    \begin{tikzpicture}[node distance=2cm, auto, scale=0.8, transform shape]
        \tikzstyle{block} = [rectangle, draw, fill=blue!20, 
            text width=8em, text centered, rounded corners, minimum height=3em]
        \tikzstyle{line} = [draw, -{Latex}]
        
        \node [block] (data) {Source de Données (\gls{bigdata})};
        \node [block, below of=data] (ingestion) {Ingestion \& Traitement (\gls{etl})};
        \node [block, below of=ingestion] (storage) {Stockage (Data Lake)};
        \node [block, below of=storage, text width=10em] (ia) {Modèle d'\gls{ai} \\ (Analyse \& Prédiction)};
        \node [block, below of=ia] (restitution) {Restitution (Dashboard)};

        \path [line] (data) -- (ingestion);
        \path [line] (ingestion) -- (storage);
        \path [line] (storage) -- (ia);
        \path [line] (ia) -- (restitution);
    \end{tikzpicture}
    \caption{Exemple de schéma d'architecture de projet de données.}
    \label{fig:archi_schema}
\end{figure}

\subsection{Hypothèse de solution}
\label{sec:hypothese_solution}
% Description de l'hypothèse de solution.
% [Votre texte ici]
%\lipsum[5] 